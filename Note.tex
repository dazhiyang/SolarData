\documentclass[twocolumn]{article}
\usepackage{lmodern}
\usepackage{amssymb,amsmath}
\usepackage{ifxetex,ifluatex}
\usepackage{fixltx2e} % provides \textsubscript
\ifnum 0\ifxetex 1\fi\ifluatex 1\fi=0 % if pdftex
  \usepackage[T1]{fontenc}
  \usepackage[utf8]{inputenc}
\else % if luatex or xelatex
  \ifxetex
    \usepackage{mathspec}
  \else
    \usepackage{fontspec}
  \fi
  \defaultfontfeatures{Ligatures=TeX,Scale=MatchLowercase}
\fi
% use upquote if available, for straight quotes in verbatim environments
\IfFileExists{upquote.sty}{\usepackage{upquote}}{}
% use microtype if available
\IfFileExists{microtype.sty}{%
\usepackage{microtype}
\UseMicrotypeSet[protrusion]{basicmath} % disable protrusion for tt fonts
}{}
\usepackage[margin=1in]{geometry}
\usepackage{hyperref}
\hypersetup{unicode=true,
            pdftitle={Note},
            pdfborder={0 0 0},
            breaklinks=true}
\urlstyle{same}  % don't use monospace font for urls
\usepackage{color}
\usepackage{fancyvrb}
\newcommand{\VerbBar}{|}
\newcommand{\VERB}{\Verb[commandchars=\\\{\}]}
\DefineVerbatimEnvironment{Highlighting}{Verbatim}{commandchars=\\\{\}}
% Add ',fontsize=\small' for more characters per line
\usepackage{framed}
\definecolor{shadecolor}{RGB}{248,248,248}
\newenvironment{Shaded}{\begin{snugshade}}{\end{snugshade}}
\newcommand{\KeywordTok}[1]{\textcolor[rgb]{0.13,0.29,0.53}{\textbf{#1}}}
\newcommand{\DataTypeTok}[1]{\textcolor[rgb]{0.13,0.29,0.53}{#1}}
\newcommand{\DecValTok}[1]{\textcolor[rgb]{0.00,0.00,0.81}{#1}}
\newcommand{\BaseNTok}[1]{\textcolor[rgb]{0.00,0.00,0.81}{#1}}
\newcommand{\FloatTok}[1]{\textcolor[rgb]{0.00,0.00,0.81}{#1}}
\newcommand{\ConstantTok}[1]{\textcolor[rgb]{0.00,0.00,0.00}{#1}}
\newcommand{\CharTok}[1]{\textcolor[rgb]{0.31,0.60,0.02}{#1}}
\newcommand{\SpecialCharTok}[1]{\textcolor[rgb]{0.00,0.00,0.00}{#1}}
\newcommand{\StringTok}[1]{\textcolor[rgb]{0.31,0.60,0.02}{#1}}
\newcommand{\VerbatimStringTok}[1]{\textcolor[rgb]{0.31,0.60,0.02}{#1}}
\newcommand{\SpecialStringTok}[1]{\textcolor[rgb]{0.31,0.60,0.02}{#1}}
\newcommand{\ImportTok}[1]{#1}
\newcommand{\CommentTok}[1]{\textcolor[rgb]{0.56,0.35,0.01}{\textit{#1}}}
\newcommand{\DocumentationTok}[1]{\textcolor[rgb]{0.56,0.35,0.01}{\textbf{\textit{#1}}}}
\newcommand{\AnnotationTok}[1]{\textcolor[rgb]{0.56,0.35,0.01}{\textbf{\textit{#1}}}}
\newcommand{\CommentVarTok}[1]{\textcolor[rgb]{0.56,0.35,0.01}{\textbf{\textit{#1}}}}
\newcommand{\OtherTok}[1]{\textcolor[rgb]{0.56,0.35,0.01}{#1}}
\newcommand{\FunctionTok}[1]{\textcolor[rgb]{0.00,0.00,0.00}{#1}}
\newcommand{\VariableTok}[1]{\textcolor[rgb]{0.00,0.00,0.00}{#1}}
\newcommand{\ControlFlowTok}[1]{\textcolor[rgb]{0.13,0.29,0.53}{\textbf{#1}}}
\newcommand{\OperatorTok}[1]{\textcolor[rgb]{0.81,0.36,0.00}{\textbf{#1}}}
\newcommand{\BuiltInTok}[1]{#1}
\newcommand{\ExtensionTok}[1]{#1}
\newcommand{\PreprocessorTok}[1]{\textcolor[rgb]{0.56,0.35,0.01}{\textit{#1}}}
\newcommand{\AttributeTok}[1]{\textcolor[rgb]{0.77,0.63,0.00}{#1}}
\newcommand{\RegionMarkerTok}[1]{#1}
\newcommand{\InformationTok}[1]{\textcolor[rgb]{0.56,0.35,0.01}{\textbf{\textit{#1}}}}
\newcommand{\WarningTok}[1]{\textcolor[rgb]{0.56,0.35,0.01}{\textbf{\textit{#1}}}}
\newcommand{\AlertTok}[1]{\textcolor[rgb]{0.94,0.16,0.16}{#1}}
\newcommand{\ErrorTok}[1]{\textcolor[rgb]{0.64,0.00,0.00}{\textbf{#1}}}
\newcommand{\NormalTok}[1]{#1}
\usepackage{graphicx,grffile}
\makeatletter
\def\maxwidth{\ifdim\Gin@nat@width>\linewidth\linewidth\else\Gin@nat@width\fi}
\def\maxheight{\ifdim\Gin@nat@height>\textheight\textheight\else\Gin@nat@height\fi}
\makeatother
% Scale images if necessary, so that they will not overflow the page
% margins by default, and it is still possible to overwrite the defaults
% using explicit options in \includegraphics[width, height, ...]{}
\setkeys{Gin}{width=\maxwidth,height=\maxheight,keepaspectratio}
\IfFileExists{parskip.sty}{%
\usepackage{parskip}
}{% else
\setlength{\parindent}{0pt}
\setlength{\parskip}{6pt plus 2pt minus 1pt}
}
\setlength{\emergencystretch}{3em}  % prevent overfull lines
\providecommand{\tightlist}{%
  \setlength{\itemsep}{0pt}\setlength{\parskip}{0pt}}
\setcounter{secnumdepth}{0}
% Redefines (sub)paragraphs to behave more like sections
\ifx\paragraph\undefined\else
\let\oldparagraph\paragraph
\renewcommand{\paragraph}[1]{\oldparagraph{#1}\mbox{}}
\fi
\ifx\subparagraph\undefined\else
\let\oldsubparagraph\subparagraph
\renewcommand{\subparagraph}[1]{\oldsubparagraph{#1}\mbox{}}
\fi

%%% Use protect on footnotes to avoid problems with footnotes in titles
\let\rmarkdownfootnote\footnote%
\def\footnote{\protect\rmarkdownfootnote}

%%% Change title format to be more compact
\usepackage{titling}

% Create subtitle command for use in maketitle
\newcommand{\subtitle}[1]{
  \posttitle{
    \begin{center}\large#1\end{center}
    }
}

\setlength{\droptitle}{-2em}
  \title{Note}
  \pretitle{\vspace{\droptitle}\centering\huge}
  \posttitle{\par}
  \author{}
  \preauthor{}\postauthor{}
  \date{}
  \predate{}\postdate{}


\begin{document}
\maketitle

\subsection{R Markdown}\label{r-markdown}

This is an R Markdown document. Markdown is a simple formatting syntax
for authoring HTML, PDF, and MS Word documents. For more details on
using R Markdown see \url{http://rmarkdown.rstudio.com}.

When you click the \textbf{Knit} button a document will be generated
that includes both content as well as the output of any embedded R code
chunks within the document. You can embed an R code chunk like this:

\begin{Shaded}
\begin{Highlighting}[]
\KeywordTok{library}\NormalTok{(SolarData)}
\end{Highlighting}
\end{Shaded}

\begin{verbatim}
## Loading required package: ggplot2
\end{verbatim}

\begin{verbatim}
## Loading required package: insol
\end{verbatim}

\begin{Shaded}
\begin{Highlighting}[]
\NormalTok{lon <-}\StringTok{ }\KeywordTok{c}\NormalTok{(}\DecValTok{100}\NormalTok{, }\OperatorTok{-}\DecValTok{100}\NormalTok{, }\DecValTok{0}\NormalTok{)}
\NormalTok{lat <-}\KeywordTok{c}\NormalTok{(}\DecValTok{10}\NormalTok{, }\DecValTok{0}\NormalTok{, }\DecValTok{90}\NormalTok{)}
\NormalTok{LT <-}\StringTok{ }\KeywordTok{LTF.get}\NormalTok{(lon, lat, }\DataTypeTok{directory =} \StringTok{"/Volumes/Macintosh Research/Completed works/Solar_Energy_2017c_Entropy/Data/Linke Turbidity"}\NormalTok{)}
\NormalTok{LT}
\end{Highlighting}
\end{Shaded}

\begin{verbatim}
##     (100, 10) (-100, 0) (0, 90)
## Jan      3.50      3.55    1.90
## Feb      3.60      3.95    1.90
## Mar      4.15      4.00    1.90
## Apr      4.45      3.80    2.00
## May      4.50      3.50    2.00
## Jun      4.55      3.65    2.05
## Jul      4.60      3.35    2.10
## Aug      4.70      3.30    2.10
## Sep      4.65      3.80    2.00
## Oct      4.40      3.75    1.95
## Nov      3.95      3.60    1.90
## Dec      3.90      3.55    1.90
\end{verbatim}

\begin{Shaded}
\begin{Highlighting}[]
\NormalTok{#####################################################################}
\NormalTok{#####################################################################}
\CommentTok{# library(SolarData)}
\CommentTok{# #California map information}
\CommentTok{# California <- maps::map('state', region = 'california', }
\CommentTok{#               fill = TRUE, plot = FALSE)}
\CommentTok{# bd <- maptools::map2SpatialPolygons(California, IDs = 'california',}
\CommentTok{#       proj4string=sp::CRS("+proj=longlat +datum=WGS84"))}
\CommentTok{# bndary <- bd@polygons[[1]]@Polygons[[1]]@coords}
\CommentTok{# bndary_plot <- data.frame(bndary)}
\CommentTok{# names(bndary_plot) <- c("lon", "lat")}
\CommentTok{# #generate the regular grid}
\CommentTok{# res = 0.2}
\CommentTok{# x1 <- seq(-124.02, -114.1, by = res)}
\CommentTok{# y1 <- seq(42.05, 32.40, by = -res)}
\CommentTok{# loc1 <- expand.grid(x1,y1) #regular grid}
\CommentTok{# loc_reg <- loc1[which(sp::point.in.polygon(point.x = loc1[,1], }
\CommentTok{#    point.y = loc1[,2], pol.x = bndary[,1], pol.y = bndary[,2])==1),]}
\CommentTok{# #generate all points following original, i.e., most granular, grid}
\CommentTok{# x2 <- seq(-124.02, -114.1, by = 0.04)}
\CommentTok{# y2 <- seq(42.05, 32.40, by = -0.04)}
\CommentTok{# loc2 <- expand.grid(x2,y2) #PSM3 grid}
\CommentTok{# loc_irreg <- loc2[which(sp::point.in.polygon(point.x = loc2[,1], }
\CommentTok{#    point.y = loc2[,2], pol.x = bndary[,1], pol.y = bndary[,2])==1),]}
\CommentTok{# loc_irreg <- loc_irreg[sample(x = 1:nrow(loc_irreg), size = }
\CommentTok{#              nrow(loc_reg), replace = FALSE),]}
\CommentTok{# }
\CommentTok{# #####################################################################}
\CommentTok{# #plot}
\CommentTok{# data_plot <- data.frame(lon = append(loc_reg[,1], loc_irreg[,1]), }
\CommentTok{#                         lat = append(loc_reg[,2], loc_irreg[,2]), }
\CommentTok{#                         group = c(rep('regular', nrow(loc_reg)), }
\CommentTok{#                         rep('irregular', nrow(loc_irreg))))}
\CommentTok{# p <- ggplot() + }
\CommentTok{#      geom_point(data=data_plot,aes(x=lon,y=lat), size = 0.5)+}
\CommentTok{#      geom_polygon(data=bndary_plot,aes(x=lon,y=lat), size = 0.5, }
\CommentTok{#                   color = 'red', fill = NA)+}
\CommentTok{#      facet_wrap(~group) +}
\CommentTok{#      coord_fixed() +}
\CommentTok{#      scale_x_continuous(limits=c(-124.5, -114), expand = c(0, 0)) +}
\CommentTok{#      scale_y_continuous(limits=c(32, 42.5), expand = c(0, 0)) +}
\CommentTok{#      xlab(expression(paste("Longitude [", degree, "]", sep = ""))) +}
\CommentTok{#      ylab(expression(paste("Latitude [", degree, "]", sep = ""))) +}
\CommentTok{#      theme_gray() +}
\CommentTok{#      theme(plot.margin = unit(c(0.5,0.5,0,0), "lines"), }
\CommentTok{#            panel.spacing = unit(0.1, "lines"), }
\CommentTok{#            text = element_text(size = 7), }
\CommentTok{#            legend.position = "none")  }
\CommentTok{# options(repr.plot.width=7, repr.plot.height=3) #This is only for Jupyter, if you want to output in pdf, using the commented code below}
\CommentTok{# p}
\end{Highlighting}
\end{Shaded}

\begin{Shaded}
\begin{Highlighting}[]
\NormalTok{#####################################################################}
\CommentTok{#install.packages("devtools", repos = "http://cran.us.r-project.org")}
\CommentTok{#library("devtools")}
\CommentTok{#install_github("dazhiyang/SolarData")}
\CommentTok{#library("SolarData") #load the package}
\CommentTok{# get PSM data for two locations}
\CommentTok{# loc <- matrix(c(42.05, 44, -124.02, -120), nrow = 2)}
\CommentTok{# PSM.get(lat = loc[,1], lon = loc[,2],}
\CommentTok{#        api_key = "FVltdchrxzBCHiSNF6M7R4ua6BFe4j81fbPp8dDP",}
\CommentTok{#        attributes = "ghi,dhi,dni",}
\CommentTok{#        name = "John+Smith", affiliation = "Some+Institute",}
\CommentTok{#        year = "2016", leap_year = "true", interval = '30',}
\CommentTok{#        utc = "false", reason_for_use = "research",}
\CommentTok{#        email = "email@gmail.com", mailing_list = "false")}
\CommentTok{# }
\CommentTok{# setwd()}
\end{Highlighting}
\end{Shaded}

\subsection{Including Plots}\label{including-plots}

Note that the \texttt{echo\ =\ FALSE} parameter was added to the code
chunk to prevent printing of the R code that generated the plot.


\end{document}
